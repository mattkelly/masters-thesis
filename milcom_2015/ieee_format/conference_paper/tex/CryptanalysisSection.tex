%------------------------------
% SECTION: Preliminary Cryptanalysis
%------------------------------
\section{Comments on Cryptanalysis}
\label{sec:Cryptanalysis}
The duplex construction has been shown to be secure against generic attacks by the \Keccak team \cite{Bertoni2011_SpongeFunctions}.
Therefore it is sufficient for us to assess only the security of the underlying sponge permutation $f$.

We provide an overview of our preliminary cryptanalysis here.
Emphasis is placed on the most general and prevalent forms of attacks.
Resistance against these techniques should result in resistance against many other less general techniques.
Our aim is to provide intuitive explanations of prevalent methods and simply explain why our permutation should be resistant.
Further cryptanalysis, as with all cryptosystems, is always welcome for future work.

%------------------------------
% Differential Cryptanalysis
%------------------------------
\subsection{Differential Cryptanalysis}
Differential cryptanalysis was publicly introduced by Biham and Shamir in 1991 in their landmark paper on the subject \cite{Biham1991_Differential}.
Since then, it has been applied with varying degrees of success to a great number of cryptosystems.
As such, it is a fundamental requirement of symmetric key cryptosystem design to prove resistance to differential cryptanalysis.

To determine the resistance of our algorithm to differential cryptanalysis, we first have to determine the maximum differential probability of our S-box.
We determined this value to be $p_{D,max} = 2^{-14}$ using an S-box evaluation program called \texttt{Eval16BitSbox} \cite{Kaminsky2014_BlockCipherAnalysis} written with Kaminsky's Parallel Java 2 library \cite{Kaminsky2014_PJ2}.

Next, it is necessary to determine the differential branch number of a round.
For this, we only need to analyze the mixer. 
We purposefully designed a mixer with differential branch number equal to three, meaning that minimally three S-boxes will be differentially active between two rounds.
This is in fact the maximum achievable branch number for a transformation defined by multiplication by a $2 \times 2$ matrix.
To verify this, we used a SAT solver called CryptoMiniSat \cite{Soos2014_CryptoMiniSat}.
This SAT solver takes as input a Boolean equation in conjunctive normal form (CNF) and determines if it is \emph{satisfiable}; that is, if it can ever produce an output of `$1$' for any set of input values.
Our CNFs were generated using Kaminsky's \texttt{SatProblem} Java class \cite{Kaminsky2014_BlockCipherAnalysis}.

The CNFs generated are unsatisfiable if and only if the mixer has differential branch number equal to three since it answers the following question: \emph{is it possible to have a difference in only one input and only one output?}
Through SAT solver analysis we determined that this is not possible for our mixer; that is, if there is a non-zero difference in only one input, there must be a non-zero difference in each output.
In the event that there is a difference in both inputs, there may be a difference in only one output.
This still leads to a differential branch number of three since two S-boxes must have been active in the previous round to lead to those two input differences.
The probability of a difference in either output is $p_{D,out} = 2^{-15}$.

With all of this information, it is possible to calculate the number of rounds needed for resistance to differential attacks.
The worst-case probability of successfully propagating a difference over two rounds is given by
\begin{equation*}
(p_{D,max})^{\mathcal{B}_D} \cdot p_{D,out},
\end{equation*}
where $\mathcal{B}_D = 3$ is the differential branch number.
Extending on this, we found that the complexity of a differential attack exceeds the complexity of a brute force attack at six rounds for a $128$-bit key.
To increase our security margin significantly, we require $10$ rounds for a $128$-bit key. 
For a $256$-bit key, $16$ rounds are required to achieve a similar security margin.

%------------------------------
% Linear Cryptanalysis
%------------------------------
\subsection{Linear Cryptanalysis}
Linear Cryptanalysis was first introduced by Matsui in 1993 in his landmark paper, \cite{Matsui1993_Linear}.
As with differential cryptanalysis, it is a fundamental requirement of symmetric key cryptosystem design to prove resistance against linear attacks.

Recall that we have verified via SAT solver analysis that the differential branch number of our mixer is three.
In \cite{Daemen2002_DesignOfRijndael}, Daemen and Rijmen prove the following result: the linear branch number of a linear transformation specified by multiplication by a matrix $M$ is equal to the differential branch number of the linear transformation specified by the transpose of that matrix.
Therefore, a sufficient condition for the differential and linear branch numbers to be equal is that the matrix is symmetric.
Our matrix is symmetric, and therefore we know $\mathcal{B}_D = \mathcal{B}_L$ without the need for further analysis.

The final step to prove the resistance of our algorithm against linear cryptanalysis involves determining the linear bias of two complete rounds of our permutation.
To combine linear biases, we use Matsui's Piling-Up Lemma from \cite{Matsui1993_Linear}:
\begin{equation*}
\epsilon = 2^{n-1} \prod\limits_{i = 1}^n \epsilon_i,
\end{equation*}
where $n = 3$ is the number of linearly active S-boxes across two rounds and $\epsilon_i = \epsilon_{L,max} = 2^{-8}$ is the worst case linear bias of those S-boxes.
Also from Matsui's paper, we know that the number of plaintext/ciphertext pairs (again referred to loosely as the \emph{complexity}) needed to exploit the overall bias $\epsilon$ is approximately $\epsilon^{-2}$.
Using this information, we determined that the complexity of a linear attack exceeds the complexity of a brute force search of a $128$-bit keyspace at six rounds.
To increase our security margin significantly, we require $10$ rounds for a $128$-bit key.
For a $256$-bit key, $16$ rounds are required to achieve a similar security margin.

\subsection{Algebraic Attacks}
Differential and linear cryptanalysis take a probabilistic approach to estimating the behavior of a system.
In contrast, algebraic attacks take a deterministic approach in that they aim to find mathematical models of a system that hold with unity probability.
For example, in 2001 Ferguson et~al.\ \cite{Ferguson2001_AlgebraicRijndael} introduced an elegant and complete algebraic representation of AES.
The ability to create such a simple mathematical representation of the cipher initially raised alarm throughout the cryptographic community.
However, the security of AES seems to be uncompromised since we believe it is far too difficult to solve such an algebraic system - the algebraic complexity of the AES S-box is too high.

Until there is reason to believe otherwise, it seems that these algebraic attacks would be highly ineffective against our permutation.
Even if there were a practical algebraic attack demonstrated on AES, which all literature indicates as highly implausible right now, the much larger size of our S-box and therefore the much higher algebraic complexity (see \cite{Wood2013_SboxThesis}) leads us to conjecture that our permutation would still be resistant.

