%-----------------------------------
% SECTION: Conclusions 
%-----------------------------------
\section{Conclusions}
In this paper we presented a customizable authenticated encryption algorithm based on the duplex construction that is targeted for hardware implementation.
We believe this algorithm to be highly secure against known attacks.
In particular, we provided proof of resistance against linear and differential attacks as well as solid reasoning for resistance against algebraic attacks.

\subsection{Future Work}
There are two primary areas for further work relating to the algorithm presented here.
As with all cryptosystems, further cryptanalysis is always appreciated.
In particular, it would be interesting to determine the best possible linear and differential trails across several rounds.

The other area of work is the hardware implementation of the algorithm described here.
In particular, quantitative results relating to the resource usage for an FPGA implementation are of great interest.
We are confident that our permutation is designed in such a way that a relatively small amount of resources will be required.

\begin{comment}
In addition to further cryptanalysis, more analysis should be performed on similar $2 \times 2$ matrices so that a list of drop-in replacements within our security margin is readily available.
It is desirable to perform statistical analysis on the remaining $16$-bit S-boxes provided by Wood.
Knowing their maximum linear biases and differential probabilities would likely allow them to act as drop-in replacements.
Until then, we have provided users with references to the tools required for such analysis.
\end{comment}

