It is common knowledge that encryption is a useful tool for providing confidentiality.
Authentication, however, is often overlooked.
Authentication provides data integrity; it helps ensure that any tampering with or corruption of data is detected.
It also provides assurance of message origin.
Authenticated encryption (AE) algorithms provide both confidentiality and integrity / authenticity by processing plaintext and producing both ciphertext and a Message Authentication Code (MAC).
It has been shown too many times throughout history that encryption without authentication is generally insecure.
This has recently culminated in a push for new authenticated encryption algorithms.

There are several authenticated encryption algorithms in existence already.
However, these algorithms are often difficult to use correctly in practice.
This is a significant problem because misusing AE constructions can result in reduced security in many cases.
Furthermore, many existing algorithms have numerous undesirable features.
For example, these algorithms often require two passes of the underlying cryptographic primitive to yield the ciphertext and MAC.
This results in a longer runtime.
It is clear that new easy-to-use, single-pass, and highly secure AE constructions are needed.
Additionally, a new AE algorithm is needed that meets stringent requirements for use in the military and government sectors.

This thesis explores the design and cryptanalysis of a novel, easily customizable AE algorithm based on the duplex construction.
Emphasis is placed on designing a secure pseudorandom permutation (PRP) for use within the construction.
A survey of state of the art cryptanalysis methods is performed and the resistance of our algorithm against such methods is considered.
The end result is an algorithm that is believed to be highly secure and that should remain secure if customizations are made within the provided guidelines.

