%------------------------------
% SECTION: Preliminary Cryptanalysis
%------------------------------
\section{Preliminary Cryptanalysis}
\label{sec:Cryptanalysis}
The duplex construction has been shown to be secure against generic attacks by the \Keccak team \cite{Bertoni2011_SpongeFunctions}.
Therefore it is sufficient for us to assess only the security of the underlying sponge permutation $f$.

We provide an overview of our preliminary cryptanalysis here.
Emphasis is placed on the most general and prevalent forms of attacks.
Resistance against these techniques should result in resistance against many other less general techniques.
Our aim is to provide intuitive explanations of prevalent methods and simply explain why our permutation should be resistant.
Further cryptanalysis, as with all cryptosystems, is always welcome for future work.

%------------------------------
% Differential Cryptanalysis
%------------------------------
\subsection{Differential Cryptanalysis}
Differential cryptanalysis was publicly introduced by Biham and Shamir in 1991 in their landmark paper on the subject \cite{Biham1991_Differential}.
Since then, it has been applied with varying degrees of success to a great number of cryptosystems.
As such, it is a fundamental requirement of symmetric key cryptosystem design to prove resistance to differential cryptanalysis.

\subsubsection{Overview}
The goal of differential cryptanalysis is to exploit non-random behavior of a system (in our case, a permutation) with regard to the propagation of differences.
A \emph{difference}, denoted $\Delta X$, is the bitwise XOR (for our case) of two bitstrings. For example,
\begin{equation*}
\Delta X = X' \oplus X''
\end{equation*}
is the difference between bitstrings $X'$ and $X''$.
For differential cryptanalysis, a difference $\Delta X$ is fed through a system and a resulting difference $\Delta Y$ is obtained.
The pair of these two related differences is called a \emph{differential} and is denoted $(\Delta X,\Delta Y)$.

Differentials occur with some associated probability.
For an ideal system the probability of a given differential is $1/2^n$ where $n$ is the length of the bitstrings involved.
A system is said to exhibit non-random behavior if the magnitude of the probability $p_D$ for some differential $(\Delta X, \Delta Y)$ is much greater than the ideal value.
This information could be used to mount an attack on the system \cite{Heys2002_Tutorial}.

To launch an effective attack, a cryptanalyst first has to focus on the S-boxes.
The S-boxes are analyzed to determine their maximum differential probabilities.

A \emph{differential trail} or \emph{characteristic} is the propagation of non-zero differentials throughout a system (i.e.\ across rounds).
A \emph{differentially active} S-box is an S-box that has a non-zero difference at its input during an attack; it is part of a differential trail.

Proving resistance against differential cryptanalysis for a substitution-permutation network requires computing a lower bound on the minimum number of active S-boxes across some number of rounds.
The more active S-boxes there are, the less likely an attack is to succeed since exponentially more chosen plaintexts will be needed for additional each active S-box.
The \emph{branch number} of an operation is of particular importance here. 
It can be simply defined as the minimum number of active S-boxes across just two rounds of a system (e.g.\ our permutation).
The technique of maximizing the branch number of a round is known as the \emph{wide trail design strategy}.
It is the main design strategy behind AES, which has a round branch number of five \cite{Daemen2001_WideTrail}\cite{Daemen2002_DesignOfRijndael}.

The number of plaintext/ciphertext pairs required to mount a successful differential attack should exceed the number required for a brute force attack.
As the differential probability reduces across rounds, more pairs are required for a successful attack.
We loosely refer to the number of pairs required as the \emph{complexity} of the differential attack.
The number of rounds is increased until this complexity exceeds that of a brute force method.

\subsubsection{Algorithm Resistance}
To determine the resistance of our algorithm to differential cryptanalysis, we first have to determine the maximum differential probability of our S-box.
We determined this value to be $p_{D,max} = 2^{-14}$ using an S-box evaluation program called \texttt{Eval16BitSbox} \cite{Kaminsky2014_BlockCipherAnalysis} written with Kaminsky's Parallel Java 2 library \cite{Kaminsky2014_PJ2}.

Next, it is necessary to determine the branch number of a round.
For this, we only need to analyze the mixer. 
We purposefully designed a mixer with differential branch number equal to three, meaning that minimally three S-boxes will be differentially active between two rounds.
This is in fact the maximum achievable branch number for a transformation defined by multiplication by a $2 \times 2$ matrix.
To verify this, we used a SAT solver called CryptoMiniSat \cite{Soos2014_CryptoMiniSat}.
This SAT solver takes as input a Boolean equation in conjunctive normal form (CNF) and determines if it is \emph{satisfiable}; that is, if it can ever produce an output of `$1$' for any set of input values.
Our CNFs were generated using Kaminsky's \texttt{SatProblem} Java class \cite{Kaminsky2014_BlockCipherAnalysis}.

The CNFs generated are unsatisfiable if and only if the mixer has differential branch number equal to three since it answers the following question: \emph{is it possible to have a difference in only one input and only one output?}
Through SAT solver analysis we determined that this is not possible for our mixer; that is, if there is a non-zero difference in only one input, there must be a non-zero difference in each output.
In the event that there is a difference in both inputs, there may be a difference in only one output.
This still leads to a differential branch number of three since two S-boxes must have been active in the previous round to lead to those two input differences.
The probability of a difference in either output is $p_{D,out} = 2^{-15}$.

With all of this information, it is possible to calculate the number of rounds needed for resistance to differential attacks.
The worst-case probability of successfully propagating a difference over two rounds is given by
\begin{equation*}
(p_{D,max})^{\mathcal{B}_D} \cdot p_{D,out},
\end{equation*}
where $\mathcal{B}_D = 3$ is the differential branch number.
From this we constructed Table~\ref{tab:DifferentialProbabilities}, which shows the worst-case differential probabilities for higher numbers of rounds.
\begin{table}[ht]
\centering
\begin{tabular}{c|c}
Rounds & Worse Case Differential Probability \\
\hline
$2$  & $2^{-57}$  \\
$4$  & $2^{-114}$ \\
$6$  & $2^{-171}$ \\
$8$  & $2^{-228}$ \\
$10$ & $2^{-285}$ \\
$12$ & $2^{-342}$ \\
$14$ & $2^{-399}$ \\
$16$ & $2^{-456}$ \\
\end{tabular}
\caption{Worst case differential probabilities over increasing rounds}
\label{tab:DifferentialProbabilities}
\end{table}

Therefore the complexity of a differential attack exceeds the complexity of a brute force search of a $128$-bit keyspace at six rounds.
To increase our security margin significantly, we require $10$ rounds for a $128$-bit key.
For a $256$-bit key, $16$ rounds are required to achieve a similar security margin.

%------------------------------
% Linear Cryptanalysis
%------------------------------
\subsection{Linear Cryptanalysis}
Linear Cryptanalysis was first introduced by Matsui in 1993 in his landmark paper, \cite{Matsui1993_Linear}.
As with differential cryptanalysis, it is a fundamental requirement of symmetric key cryptosystem design to prove resistance against linear attacks.

\subsubsection{Overview}
Linear cryptanalysis is surprisingly similar to differential cryptanalysis in many ways. 
However, for linear cryptanalysis we are concerned with estimating the behavior of a system using linear expressions rather than highly probable differential characteristics.
As with differential cryptanalysis, the first step is to analyze the S-boxes involved in the substitution-permutation network.
An S-box, by definition, should be highly nonlinear to provide sufficient confusion. 
However, it is possible to uncover linear approximations of S-box outputs that occur with high (or low) probability.
We can represent our S-box as a vectorial Boolean function 
\begin{equation*}
S \from \Ztwo^{16} \to \Ztwo^{16}
\end{equation*}
in which the input $X$ and output $Y$ are represented as row vectors, e.g.\
\begin{equation*}
X = \begin{pmatrix}X_1 & X_2 & ... & X_{15} & X_{16} \end{pmatrix},
\end{equation*}
where $X_i \in \Ztwo$.
Favoring typical convention found in the literature, $X_1$ is the MSB \cite{Heys2002_Tutorial}.
This notation allows us to easily represent linear approximations in the form
\begin{equation*}
\left( \bigoplus\limits_{i=1}^{16} X_i \right) = \left( \bigoplus\limits_{i=1}^{16} Y_i \right).
\end{equation*}

The ideal \emph{linear probability} $p$ that such an approximation holds true is exactly equal to 1/2.
We are concerned with deviations from this ideal probability, known as the \emph{linear bias}, $\epsilon$. 
Clearly, $\epsilon = p - 1/2$.
We found the maximum linear bias of our particular S-box to be $\epsilon_{L,max} = 2^{-8}$ using the same program as previously described.

The \emph{linear branch number} $\mathcal{B}_L$ is the minimum number of linearly active S-boxes across two rounds of our permutation.
As with differential cryptanalysis, it depends solely on our mixer.

\subsubsection{Algorithm Resistance}
Recall that we have verified via SAT solver analysis that the differential branch number of our mixer is three.
In \cite{Daemen2002_DesignOfRijndael}, Daemen and Rijmen prove the following result: the linear branch number of a linear transformation specified by multiplication by a matrix $M$ is equal to the differential branch number of the linear transformation specified by the transpose of that matrix.
Therefore, a sufficient condition for the differential and linear branch numbers to be equal is that the matrix is symmetric.
Our matrix is symmetric, and therefore we know $\mathcal{B}_D = \mathcal{B}_L$ without the need for further analysis.

The final step to prove the resistance of our algorithm against linear cryptanalysis involves determining the linear bias of two complete rounds of our permutation.
To combine linear biases, we use Matsui's Piling-Up Lemma from \cite{Matsui1993_Linear}:
\begin{equation*}
\epsilon = 2^{n-1} \prod\limits_{i = 1}^n \epsilon_i,
\end{equation*}
where $n = 3$ is the number of linearly active S-boxes across two rounds and $\epsilon_i = \epsilon_{L,max} = 2^{-8}$ is the worst case linear bias of those S-boxes.
Note that we need not consider the mixer since, being a linear function, it must have a maximum linear probability $p_{mix} = 1$, corresponding to a maximum bias of $\epsilon_{mix} = 1/2$. This gets cancelled out.
Also from Matsui's paper, we know that the number of plaintext/ciphertext pairs (again referred to loosely as the \emph{complexity}) needed to exploit the overall bias $\epsilon$ is approximately $\epsilon^{-2}$.
Using this information, we constructed Table~\ref{tab:LinearBiases}.

\begin{table}[ht]
\centering
\begin{tabular}{c|c|c}
Rounds & Worst Case Linear Bias & PT/CT Pairs Required \\
\hline
$2$  & $2^{-22}$  & $2^{44}$  \\
$4$  & $2^{-44}$  & $2^{88}$  \\
$6$  & $2^{-66}$  & $2^{132}$ \\
$8$  & $2^{-88}$  & $2^{176}$ \\
$10$ & $2^{-110}$ & $2^{220}$ \\
$12$ & $2^{-132}$ & $2^{264}$ \\
$14$ & $2^{-154}$ & $2^{308}$ \\
$16$ & $2^{-176}$ & $2^{352}$ \\
\end{tabular}
\caption{Worst case linear biases and linear attack complexities over increasing rounds}
\label{tab:LinearBiases}
\end{table}

Therefore the complexity of a linear attack exceeds the complexity of a brute force search of a $128$-bit keyspace at six rounds.
To increase our security margin significantly, we require $10$ rounds for a $128$-bit key.
For a $256$-bit key, $16$ rounds are required to achieve a similar security margin.

\subsection{Algebraic Attacks}
Differential and linear cryptanalysis take a probabilistic approach to estimating the behavior of a system.
In contrast, algebraic attacks take a deterministic approach in that they aim to find mathematical models of a system that hold with unity probability.
For example, in 2001 Ferguson et~al.\ \cite{Ferguson2001_AlgebraicRijndael} introduced an elegant and complete algebraic representation of AES.
The ability to create such a simple mathematical representation of the cipher initially raised alarm throughout the cryptographic community.
However, the security of AES seems to be uncompromised since we believe it is far too difficult to solve such an algebraic system - the algebraic complexity of the AES S-box is too high.

Until there is reason to believe otherwise, it seems that these algebraic attacks would be highly ineffective against our permutation.
Even if there were a practical algebraic attack demonstrated on AES, which all literature indicates as highly implausible right now, the much larger size of our S-box and therefore the much higher algebraic complexity (see \cite{Wood2013_SboxThesis}) leads us to conjecture that our permutation would still be resistant.

